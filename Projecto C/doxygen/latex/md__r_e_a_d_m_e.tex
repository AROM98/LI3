Projeto de L\+I3 -\/ 2018/2019

Sistema de Gestão das Vendas de uma Distribuidora com 3 Filiais

Introdução e Objectivos.

O projecto de C da disciplina de L\+I3 de M\+I\+EI tem por objectivo fundamental ajudar à consolidação experimental dos conhecimentos teóricos e práticos adquiridos nas U\+Cs anteriores e a introdução de novos conceitos. Os objectivos de L\+I3, e deste trabalho, não se restringem apenas a aumentar os conhecimentos dos alunos na linguagem C, o que seria até questionável, mas, e talvez fundamentalmente, apresentar aos alunos de forma pragmática, os desafios que se colocam a quem concebe e programa aplicações software (em qualquer linguagem), quando passamos a realizar a designada programação em larga escala, ou seja, aplicações com grandes volumes de dados e com mais elevada complexidade algorítmica e estrutural. De facto, quando passamos para tais patamares de complexidade, torna-\/se imperioso conhecer e usar os melhores princípios da Engenharia de Software, de modo a que tais projectos de software, em geral realizados por equipas, possam ser concebidos com melhor estrutura, de modo a que sejam mais facilmente modificáveis, e sejam, apesar da complexidade, o mais optimizados possível a todos os níveis. Para que tal seja possível teremos que introduzir novos princípios de programação, mais adequados à programação em grande escala, designadamente\+: -\/$>$ Modularidade e encapsulamento de dados usando construções da linguagem; -\/$>$ Criação de código reutilizável; -\/$>$ Escolha optimizada das estruturas de dados e reutilização; -\/$>$ Testes de performance e até profiling. Este projecto, a desenvolver em trabalho de grupo (de no máximo 3 alunos), visa a experimentação e aplicação destas práticas de desenvolvimento de software usando a linguagem C, práticas que são extensíveis a outras linguagens e paradigmas. 